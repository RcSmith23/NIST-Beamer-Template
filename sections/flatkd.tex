
\begin{frame}
  \frametitle{Flattened \kd trees}
  \framesubtitle{Introduction}

  \begin{itemize}
    \item Flattened \kd trees, similar to traditional \kd trees, partition space using hyperplanes
      that are fixed w.r.t one dimension of the original space
    \item Flattened trees take a parameter $b$, the branching factor which must be a power of two, and
      organize $b-1$ hyperplanes into a single node in a BST
    \item Each node has up to $b$ children, each of which accounts for a disjoint subspace of the parent's
      space
  \end{itemize}

\end{frame}

\begin{frame}
  \frametitle{Flattened \kd trees}
  \framesubtitle{Example: $k=3$ and $b=8$}

  insert graphix

\end{frame}

\begin{frame}
  \frametitle{Flattened \kd trees}
  \framesubtitle{Advantages}

  \begin{itemize}
    \item Increased dimensionality of nodes significantly reduces height of tree
    \item Closely associated hyperplanes allow for more powerful look ahead to adjacent
      spaces during nearest neighbor search
  \end{itemize}

\end{frame}

\begin{frame}
  \frametitle{Flattened \kd trees}
  \framesubtitle{Disadvantages}

  \begin{itemize}
    \item Additional computation required at each node during search
    \item Grouped hyperplanes increase memory footprint of nodes 
    \item More complex conditions during nearest neighbor adjacent subspace
      checks
  \end{itemize}

\end{frame}

\begin{frame}
  \frametitle{Flattened \kd trees}
  \framesubtitle{NN adjaceny check}

  Insert image displaying additional computation for adjaceny check

\end{frame}

\begin{frame}
  \frametitle{Flattened \kd trees}
  \framesubtitle{Initial results}

  show run performance results (time, dereferences, total nodes)

\end{frame}

\begin{frame}
  \frametitle{Flattened \kd trees}
  \framesubtitle{Initial results}

  show performance results for cache usage (instruction count, cache misses)

\end{frame}

\begin{frame}
  \frametitle{Flattened \kd trees}
  \framesubtitle{Shortcomings}

  \begin{itemize}
    \item 
  \end{itemize}

\end{frame}
